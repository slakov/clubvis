\documentclass{article}
\usepackage{amsmath}
\usepackage{amssymb}
\usepackage{graphicx}
\usepackage{hyperref}

\title{Mathematical Model of Club Membership Dynamics}
\author{ClubViz Documentation}
\date{\today}

\begin{document}
\maketitle

\section{Model Overview}
The club membership simulation models the dynamics of trait-based segregation in multiple clubs. The model consists of:
\begin{itemize}
    \item A population of $N$ individuals with two traits (B and R)
    \item $C$ independent clubs
    \item Time-discrete Markov chain dynamics
\end{itemize}

\section{Core Parameters}
\begin{itemize}
    \item $N$: Total population size
    \item $C$: Number of clubs
    \item $k$: Base join probability per turn
    \item $t$: Leave threshold (proportion of B trait)
    \item $p_{high}$: High leave probability (when trait is underrepresented)
    \item $p_{low}$: Low leave probability (when trait is well-represented)
    \item $r_{pop}$: Proportion of R trait in population
    \item $b_{pop}$: Proportion of B trait in population ($1 - r_{pop}$)
\end{itemize}

\section{Markov Chain Model}
Each club operates as an independent Markov chain. The state of a club at time $n$ is characterized by:
\begin{itemize}
    \item $B_n$: Number of B-trait members
    \item $R_n$: Number of R-trait members
    \item $T_n = B_n + R_n$: Total members
\end{itemize}

\subsection{Transition Probabilities}
For each person and club in each turn:

\subsubsection{Joining Process}
\begin{itemize}
    \item If person is not a member:
    \[ P(\text{join}) = \frac{k}{C} \]
    \item Join probability is independent of trait
\end{itemize}

\subsubsection{Leaving Process}
\begin{itemize}
    \item If person is a member and didn't just join:
    \[ P(\text{leave}) = \begin{cases}
        p_{high} & \text{if trait is underrepresented} \\
        p_{low} & \text{if trait is well-represented}
    \end{cases} \]
    \item Underrepresentation is determined by:
    \[ \text{is\_underrepresented} = \begin{cases}
        \text{true} & \text{if } \frac{B_n}{T_n} < t \text{ for B trait} \\
        \text{true} & \text{if } \frac{R_n}{T_n} < (1-t) \text{ for R trait}
    \end{cases} \]
\end{itemize}

\section{Equilibrium Analysis}
At equilibrium, the flow of members in and out of clubs balances for each trait.

\subsection{Membership Rates}
For B-trait individuals:
\[ \text{membership\_rate}_B = \frac{k/C}{(k/C) + p_B} \]
where $p_B$ is the effective leave probability for B-trait members.

For R-trait individuals:
\[ \text{membership\_rate}_R = \frac{k/C}{(k/C) + p_R} \]
where $p_R$ is the effective leave probability for R-trait members.

\subsection{Club Composition}
The proportion of B in a club at equilibrium:
\[ p = \frac{\text{membership\_rate}_B \cdot b_{pop}}{\text{membership\_rate}_B \cdot b_{pop} + \text{membership\_rate}_R \cdot r_{pop}} \]

\subsection{Equilibrium Points}
Two equilibrium points exist:

1. Lower Equilibrium ($p < t$):
\[ p_{lower} = \frac{\frac{k/C}{k/C + p_{high}} \cdot b_{pop}}{\frac{k/C}{k/C + p_{high}} \cdot b_{pop} + \frac{k/C}{k/C + p_{low}} \cdot r_{pop}} \]

2. Upper Equilibrium ($p > t$):
\[ p_{upper} = \frac{\frac{k/C}{k/C + p_{low}} \cdot b_{pop}}{\frac{k/C}{k/C + p_{low}} \cdot b_{pop} + \frac{k/C}{k/C + p_{high}} \cdot r_{pop}} \]

\section{Stability Analysis}
The stability of equilibrium points depends on:

\subsection{Barrier Height}
\[ \text{barrier\_height} = \min(t - p_{lower}, p_{upper} - t) \cdot (p_{high} - p_{low}) \]

\subsection{Stochastic Fluctuations}
\[ \text{fluctuation\_factor} = \frac{k}{C} + \frac{p_{high} + p_{low}}{2} \]

\subsection{Transition Likelihood}
\[ \text{transition\_likelihood} = \frac{\text{fluctuation\_factor}}{\text{barrier\_height}} \]

\section{Key Properties}
\begin{itemize}
    \item \textbf{Independence}: Each club operates as an independent Markov chain
    \item \textbf{Bistability}: Two stable equilibrium points exist
    \item \textbf{Threshold Dependence}: Leave probabilities depend on trait representation relative to threshold
    \item \textbf{Population Size Effects}: Smaller populations experience larger relative fluctuations
\end{itemize}

\section{Parameter Effects}
\begin{itemize}
    \item \textbf{Increasing $k$}: Higher join rates create more turnover
    \item \textbf{Decreasing $C$}: Fewer clubs increase effective join rate per club
    \item \textbf{Reducing $(p_{high} - p_{low})$}: Weaker feedback mechanism
    \item \textbf{Balancing $r_{pop}$ and $b_{pop}$}: More equal representation
\end{itemize}

\section{Simulation Behavior}
The simulation typically exhibits:
\begin{itemize}
    \item Initial rapid changes in club composition
    \item Stabilization around equilibrium points
    \item Occasional transitions between equilibria
    \item Stochastic fluctuations around stable points
\end{itemize}

\end{document} 